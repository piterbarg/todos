
\documentclass{article}
%%%%%%%%%%%%%%%%%%%%%%%%%%%%%%%%%%%%%%%%%%%%%%%%%%%%%%%%%%%%%%%%%%%%%%%%%%%%%%%%%%%%%%%%%%%%%%%%%%%%%%%%%%%%%%%%%%%%%%%%%%%%%%%%%%%%%%%%%%%%%%%%%%%%%%%%%%%%%%%%%%%%%%%%%%%%%%%%%%%%%%%%%%%%%%%%%%%%%%%%%%%%%%%%%%%%%%%%%%%%%%%%%%%%%%%%%%%%%%%%%%%%%%%%%%%%
\usepackage{amssymb}
\usepackage{amsfonts}
\usepackage{amsmath}
\usepackage{graphicx}

\setcounter{MaxMatrixCols}{10}
%TCIDATA{OutputFilter=LATEX.DLL}
%TCIDATA{Version=5.50.0.2953}
%TCIDATA{<META NAME="SaveForMode" CONTENT="1">}
%TCIDATA{BibliographyScheme=Manual}
%TCIDATA{Created=Tuesday, February 04, 2020 09:00:51}
%TCIDATA{LastRevised=Wednesday, March 11, 2020 11:28:41}
%TCIDATA{<META NAME="GraphicsSave" CONTENT="32">}
%TCIDATA{<META NAME="DocumentShell" CONTENT="Standard LaTeX\Standard LaTeX Article">}
%TCIDATA{Language=American English}
%TCIDATA{CSTFile=40 LaTeX article.cst}

\newtheorem{theorem}{Theorem}
\newtheorem{acknowledgment}[theorem]{Acknowledgment}
\newtheorem{algorithm}[theorem]{Algorithm}
\newtheorem{axiom}[theorem]{Axiom}
\newtheorem{case}[theorem]{Case}
\newtheorem{claim}[theorem]{Claim}
\newtheorem{conclusion}[theorem]{Conclusion}
\newtheorem{condition}[theorem]{Condition}
\newtheorem{conjecture}[theorem]{Conjecture}
\newtheorem{corollary}[theorem]{Corollary}
\newtheorem{criterion}[theorem]{Criterion}
\newtheorem{definition}[theorem]{Definition}
\newtheorem{example}[theorem]{Example}
\newtheorem{exercise}[theorem]{Exercise}
\newtheorem{lemma}[theorem]{Lemma}
\newtheorem{notation}[theorem]{Notation}
\newtheorem{problem}[theorem]{Problem}
\newtheorem{proposition}[theorem]{Proposition}
\newtheorem{remark}[theorem]{Remark}
\newtheorem{solution}[theorem]{Solution}
\newtheorem{summary}[theorem]{Summary}
\newenvironment{proof}[1][Proof]{\noindent\textbf{#1.} }{\ \rule{0.5em}{0.5em}}
\input{tcilatex}
\begin{document}

\title{Interest Rates Benchmark Reform and Options Markets}
\author{Vladimir V. Piterbarg \\
%EndAName
NatWest Markets}
\maketitle

\begin{abstract}
We examine the impact of interest rates benchmark reform and upcoming Libor
transition on options markets. We address various modelling challenges the
transition brings. We specifically focus on the impact of the clearing
houses' discounting switch on swaptions, and the consequences of Libor
transition on Libor-in-arrears swaps, caps, and range accruals as typical
representatives of a very wide range of Libor derivatives.
\end{abstract}

\section{Introduction}

Fundamental changes in market structure due to the impending Libor
transition, and more generally interest rate benchmark reform, are in the
forefront of the minds of market participants. Close attention and
voluminous discourse is afforded to various topics such as proposed
fallbacks and impact on bilateral trading \cite{risk-mh1}, pre-cessation
triggers \cite{risk-presess}, discounting changes by clearing houses (CCPs) 
\cite{risk-swpt}, impact on cash versus derivatives markets \cite{risk-bonds}%
, and many others. The main focus of collective attention so far is firmly
on linear markets such as swaps, bonds and loans. Potential impact on
non-linear markets has received relatively less attention. In this paper we
examine the impact of the benchmark reform on options, or more generally
non-linear rates markets and highlight significant challenges for certain
product types. We make a number of suggestions how these challenges could be
addressed.

We focus on options markets in USD, EUR and GBP as being the most liquid and
the furthest along in the benchmark reform. The three markets are reasonably
similar, yet exhibit important idiosyncrasies in conventions and benchmark
reform approaches. Some of the considerations below are more important for
some of these markets and less for the others, and we endeavor to highlight
such differences as we go along.

As a small caveat, market developments in the rate benchmark reform space
come at a fast pace and while the information here is believed to be
accurate at the time of writing, some later developments may not be
reflected.

\section{Discounting and Swaptions}

The impact of the impending CCP discounting switch on swaptions has probably
received the most attention as far as non-linear markets are concerned, see
e.g.~\cite{risk-swpt} and \cite{arrc}. Let us quickly review the issue. As
part of rates benchmark reform, a number of CCPs (e.g.\ LCH, CME) have
announced that their collateral (so-called PAI, or Price Alignment Interest)
rates will switch from legacy overnight rates to their replacements, such as
from FedFunds to SOFR for USD and from Eonia to ESTR for EUR, targeting
October (USD) and June (EUR) 2020 as the implementation date. Rates used to
fund collateral balances directly define rates for discounting cashflows in
cleared instruments (see \cite{vp-disc}). Values (and risk sensitivities)\
of cleared interest rate swaps will change on that date. As part of the
transition, CCPs have proposed a compensation mechanism to eliminate the
potential value transfer.

Unlike swaps, swaptions -- options to enter swaps -- are not cleared and are
governed by bilateral agreements based on ISDA templates. In USD and EUR
swaptions typically settle into cleared swaps or, more accurately, into cash
amounts determined by referencing cleared swap values\footnote{%
The situation in EUR markets is in fact even more subtle as swaptions use
ICE screen rate for settlement which at the moment is not CCP-specific and
is not linked to DV01 referenced in a contract. We do not explore this
important nuance here.}. If the CCP discounting switch happens before
swaption expiry, the underlying value of the referenced swap will change
resulting in a value transfer in a bilateral transaction.

Establishing a market-wide compensation scheme for the value transfer in
swaptions is a complex issue, as \cite{risk-swpt} outlines. ISDA and various
regulators are reportedly examining possible options, see e.g.~\cite{arrc},
and even some fintechs are making proposals. The markets are, reportedly,
skeptical (arguably more in EUR than in USD) that an effective solution can
be found or that, at the very least, swaptions traded post discounting
switch announcement date would be in scope (see more on this in Section \ref%
{sec:swpt_vols}). Some market participants (in EUR; not so much in USD) have
been showing prices on packages of synthetic swap exposures via CCP-settled
zero-wide collars (off-market i.e.\ with strikes at ATM+$Y$ with $Y$ being
50 or even 100 bp) hedged by cleared swaps, positions that should be worth
exactly zero if the same discounting applied to all legs. Such a package
expresses a view that a cleared swap will be compensated for the value
transfer but the swaption position will not be, the difference maximized for
deep in-the-money swaps (hence strikes at ATM+50 to 100 bp).

We emphasize that two market wide event dates are important, the date(s)
when CCP(s) announced the intent to switch discounting for cleared swaps
(September 2019) and the actual discounting switch date (June 2020 for EUR,
October 2020 for USD). For a particular swaption two dates are important in
our context, the trade date and the expiry date.

\subsection{Compensation for Value Transfer\label{sec:swpt_comp}}

Leaving the issue of how to best persuade beneficiaries of discounting
switch value transfer in swaptions to compensate their counterparts, we
consider a simpler question of what the right compensation should be.

The proposal for value transfer compensation for swaps is fairly
straightforward, see \cite{cme-ds}. As the discounting rate changes, two
things happen. The value of a swap changes, and it now has sensitivities
(deltas) to the new discounting rate and not the old one. The first one is
addressed via a cash payment, and the second, where required, via an
exchange of basis swaps. A clearing house, being in the middle of all
cleared swaps, facilitates all this activity.

It is tempting to suggest that a similar compensation mechanism for
swaptions, i.e.\ an exchange of a package of cash and basis swaps at
swaption expiry is adopted to negate the value and risk transfer. While
compellingly simple -- the CCP\ mechanism for swaps is applied to swaptions
with no modifications -- there are significant issues with this approach.
The main one, in our view, is the fact that the swaption expiry may well be
20 or 30 years from now, and it is entirely possible (in fact, expected)\
that the legacy OIS rates/curves (specifically FedFunds; Eonia has been
fixed to be ESTR$+8.5$ bp which, conceivably, could still be used in the far
future) will no longer be available. Moreover, valuing and risk-managing a
swaption with such compensation mechanism throughout its life will be a
considerable challenge. Since at expiry one would effectively exercise not
into a vanilla swap but a basket of a swap, cash, and basis swaps, the
swaption will cease to be a ``simple'' vanilla option anymore.

It should be clear that cash compensation at the time of the discounting
switch (or announcement of discounting switch which is now sadly in the
past)\ is a better option and if fact is one of the proposed solutions in
the consultation paper \cite{arrc}. We argue that a more theoretically-sound
mechanism is possible, that we now proceed to outline.

Discounting swaptions generally requires two curves. The curve specified by
the bilateral CSA is used for discounting of the swaption payoff from
swaption expiry date to today. The CCP discounting curve is used for the
underlying swap in payoff calculations. We focus on the CCP discounting
switch and remove the bilateral discounting curve from consideration by
focusing on forward (to swaption expiry) quantities.

Let $T$ be the expiry date of the swaption. We fix a tenor structure $%
0<T=T_{1}<\dots <T_{N},$ $\tau _{i}=T_{i+1}-T_{i}$. Let $%
L_{i}(t)=L(t,T_{i},T_{i+1})$ be Libor rates, $%
P_{i}^{f}(t)=P^{f}(t,T,T_{i})=P^{f}(t,T_{i})/P^{f}(t,T)$ the
\textquotedblleft old\textquotedblright\ OIS, e.g.~FedFunds, forward
discount factors, and $%
P_{i}^{s}(t)=P^{s}(t,T,T_{i})=P^{s}(t,T_{i})/P^{s}(t,T)$ the
\textquotedblleft new\textquotedblright , say SOFR, forward discount
factors. Let us denote the forward annuity and the swap rate under the two
different discount curves by%
\begin{align*}
A^{f,s}(t)& =\sum_{i}\tau _{i}P_{i+1}^{f,s}(t), \\
S^{f,s}(t)& =\frac{1}{A^{f,s}(t)}\sum_{i}\tau _{i}P_{i+1}^{f,s}(t)L_{i}(t).
\end{align*}%
The (forward to time $T$) value of a swaption to exercise into a FedFunds
(SOFR) discounted CCP swap (technically its CCP cash value, but the
distinction is irrelevant for modelling) with strikes $K^{f}$($K^{s}$) and
notionals $N^{f}$($N^{s})$ is 
\begin{equation*}
V_{\mathrm{swpt}}^{f,s}(t)=N^{f,s}A^{f,s}(t)\mathrm{E}%
_{t}^{A^{f,s}}(S^{f,s}(T)-K^{f,s})^{+}.
\end{equation*}%
Let us suppose a mechanism could be agreed to modify contractual terms of a
swaption (notional, strike) due to the discounting switch. The potential
value transfer is then given by 
\begin{equation*}
V_{\mathrm{swpt}}^{f}(t)-V_{\mathrm{swpt}}^{s}(t)=N^{f}A^{f}(t)\mathrm{E}%
_{t}^{A^{f}}(S^{f}(T)-K^{f})^{+}-N^{s}A^{s}(t)\mathrm{E}%
_{t}^{A^{s}}(S^{s}(T)-K^{s})^{+}.
\end{equation*}%
Using the same FedFunds annuity measure for both expressions we obtain%
\begin{align*}
\Delta V_{\mathrm{swpt}}& \triangleq V_{\mathrm{swpt}}^{f}(t)-V_{\mathrm{swpt%
}}^{s}(t) \\
& =N^{f}A^{f}(t)\mathrm{E}_{t}^{A^{f}}\left( (S^{f}(T)-K^{f})^{+}-c\frac{%
A^{s}(T)}{A^{f}(T)}\left( S^{s}(T)-K^{s}\right) ^{+}\right) ,
\end{align*}%
where we have denoted%
\begin{equation}
c=\frac{N^{s}}{N^{f}}.  \label{eq:swpt1}
\end{equation}

Our objective is to find the change in contractual terms that would
minimize/eliminate value transfer. Clearly the first thing we would want to
do is to match the moneyness of the old and the new swaption. To that effect
we set%
\begin{equation}
K^{s}=K^{f}+S^{s}(t)-S^{f}(t),  \label{eq:swpt4}
\end{equation}%
i.e.\ we simply adjust the swaption strike by the difference in swap rates
under the old and the new discounting. We do not expect the strike
adjustment to be large as discounting has small effect on at-the-money
swaps. (In fact is some cases the strike adjustment can be dispensed with
altogether.) Let us simplify the notations and set

\begin{equation*}
S(T)-K\triangleq S^{f}(T)-K^{f}\approx S^{s}(T)-K^{s}
\end{equation*}%
where the last equality follows from the strike adjustment (\ref{eq:swpt4})
and some obvious assumptions. With this step completed, the expression for
the potential value transfer reduces to%
\begin{equation*}
\Delta V_{\mathrm{swpt}}=N^{f}A^{f}(t)\mathrm{E}_{t}^{A^{f}}\left( \left( 1-c%
\frac{A^{s}(T)}{A^{f}(T)}\right) \left( S(T)-K\right) ^{+}\right) .
\end{equation*}

Clearly $\frac{A^{s}(T)}{A^{f}(T)}$ is a positive\ martingale under $\mathrm{%
P}^{A^{f}}$,%
\begin{equation*}
\frac{A^{s}(T)}{A^{f}(T)}=\frac{A^{s}(t)}{A^{f}(t)}M(T),\quad M(t)=1,\text{%
\quad }M(\cdot )\text{ is a }\mathrm{P}^{A^{f}}\text{-martingale.}
\end{equation*}%
Setting the value transfer to zero, 
\begin{equation*}
0=A^{f}(t)\mathrm{E}_{t}^{A^{f}}\left( \left( 1-c\frac{A^{s}(t)}{A^{f}(t)}%
M(T)\right) (S(T)-K)^{+}\right) ,
\end{equation*}%
we obtain the following condition on $c$,%
\begin{equation*}
A^{f}(t)\left( c\frac{A^{s}(t)}{A^{f}(t)}\right) \mathrm{E}%
_{t}^{A^{f}}\left( M(T)(S(T)-K)^{+}\right) =A^{f}(t)\mathrm{E}%
_{t}^{A^{f}}\left( (S(T)-K)^{+}\right) .
\end{equation*}%
The ratio of notionals (\ref{eq:swpt1}) that is consistent with zero value
transfer (under our assumptions) is then%
\begin{equation}
c=\frac{A^{f}(t)}{A^{s}(t)}\times \frac{\mathrm{E}_{t}^{A^{f}}\left(
(S(T)-K)^{+}\right) }{\mathrm{E}_{t}^{A^{f}}\left( M(T)(S(T)-K)^{+}\right) }.
\label{eq:swpt2}
\end{equation}%
The factor 
\begin{equation}
c\approx \frac{A^{f}(t)}{A^{s}(t)}  \label{eq:c1}
\end{equation}%
is the dominant one; it specifies that we should replace one FedFunds
swaption with $c$ SOFR swaptions, where $c$ is the ratio of the FedFunds and
SOFR annuities of the underlying swap or, simply, forward DV01s (forward
parallel deltas) of the old and the new swaption.

The second factor in (\ref{eq:swpt2}) is a convexity correction to the basic
expression for $c$ in (\ref{eq:c1}) and is driven by the correlation between 
$M(T)$ and $S(T)$ or, more accurately, $(S(T)-K)^{+}$ and as such is strike
dependent. To first order, it can be ignored. If more precision is required,
the following calculation can be employed,%
\begin{align}
\mathrm{E}_{t}^{A^{f}}\left( M(T)(S(T)-K)^{+}\right) & =\mathrm{E}%
_{t}^{A^{f}}\left( \mathrm{E}_{t}^{A^{f}}\left( \left.
M(T)(S(T)-K)^{+}\right\vert S(T)\right) \right)  \notag \\
& =\mathrm{E}_{t}^{A^{f}}\left( (S(T)-K)^{+}\mathrm{E}_{t}^{A^{f}}\left(
\left. M(T)\right\vert S(T)\right) \right)  \notag \\
& =\mathrm{E}_{t}^{A^{f}}\left( (S(T)-K)^{+}\mu (S(T)\right) ,
\label{eq:ca1}
\end{align}%
where%
\begin{equation*}
\mu (x)\triangleq \mathrm{E}_{t}^{A^{f}}\left( \left. M(T)\right\vert
S(T)=x\right) \approx \frac{\left\langle M(T),S(T)\right\rangle }{%
\left\langle S(T),S(T)\right\rangle }x.
\end{equation*}%
The adjustment in (\ref{eq:swpt2}) vs.~(\ref{eq:c1}) can then be calculated
by strike replication of the payoff (\ref{eq:ca1}) with European swaptions
in a way not dissimilar to CMS convexity calculations, see \cite{ap-book}.

It could be argued that the convexity correction is not needed as the ratio
of discount annuities is only weakly correlated to the swap rate; on the
other hand, in the particular case of FedFunds vs.~SOFR, the former
incorporates some credit risk premium while the latter does not, and $%
S(\cdot )$ also has certain amount of credit risk premium coming from Libor,
so the correlation is probably not zero.

To summarize, we propose the following adjustments to bilateral in-scope
CCP-settled swaptions when CCPs change discounting:

\begin{itemize}
\item Replace one unit of $f$ swaption with $c$ units of $s$ swaptions,
where $c$ is calculated as the ratio of forward annuities (or more
accurately including the convexity correction);

\item Adjust the strike of the swaption by the difference of the swap rates
under FedFunds and SOFR discounting at the time of conversion.
\end{itemize}

This approach, in our view, enjoys a number of advantages over a simple cash
adjustment on the discounting switch date (and certainly over the largely
unworkable adjustment at the swaption expiry):

\begin{itemize}
\item Deltas to the Libor (projection) curve remain largely unchanged so no
extra hedging is required;

\item Deltas to discounting curves are comparable in the sense that $X$
units of risk to FedFunds are replaced by $X$ units of risk to SOFR;

\item Vegas (volatility sensitivities) remain largely the same, not
requiring any additional trading in options to recover the original vega
position as would be the case under the cash compensation scheme.
\end{itemize}

It is fair to note that our approach comes with its own limitations.
Arguably, it is more practical to exchange cash as compensation at a point
in time than to amend a potentially large number of existing trades. Perhaps
a hybrid approach is warranted with market participants agreeing on which
route balances their risk management and operational considerations best.

The discounting change is only one of the aspects of the benchmark reform
that will affect swaptions markets. Down the line, when Libor ceases to
exist, swaptions underlyings will change from Libor to OIS swaps (except in
EUR). Potentially, this may lead to other considerations related to value
transfer. We do not consider these issues here as there continues to be a
fair amount of uncertainty over important details of this transition. At
this moment, however, we do not expect a significant valuation impact as the
Libor-OIS basis spreads have already largely converged to their proposed
fallback levels, see \cite{henrard-ssrn1}, reducing or potentially
eliminating future reform-related valuation changes.

\subsection{Volatility Smile Impact\label{sec:swpt_vols}}

Swaptions are quoted in price terms (premiums or, more commonly, forward
premiums) in the inter-dealer markets that all dealers use for price
discovery. Internally, most dealers represent swaption prices in terms of
implied volatilities so that, when inserted into the Black-Scholes (or
Normal) model and the values scaled by DV01s, market-observed prices are
recovered. Implied volatilities are then used to mark swaption books
internally, determine swaption prices offered to clients, and also feed into
models for more exotic products such as CMS, Bermudan swaptions, and so on.

Consider a dealer who, pre discounting switch announcement, marks a
particular swaption expiring post switch (with notations from Section \ref%
{sec:swpt_comp}) at volatility $v^{f}$ that corresponds to value $V_{\mathrm{%
swpt}}^{f}(t)$, swap rate $S^{f}(t),$ strike $K^{f},$ and annuity $A^{f}(t).$
The question we consider is where should he mark this volatility
post-announcement. There is a range of plausible scenarios.

\subsubsection{Expectation of No Value Transfer}

Let us consider the case where the market believes that parties to a
pre-announcement, or ``legacy'', swaption will be compensated. In fact let
us assume that the market expects compensation along the lines of Section %
\ref{sec:swpt_comp}. As there is no expectation of value transfer, there is
no jump in the price $V_{\mathrm{swpt}}^{f}(t)$. The dealer should continue
using existing valuation parameters, i.e.\ the forward rate $S^{f}(t),$
strike $K^{f},$ FedFunds for discounting the underlying swap (i.e.\ $A^{f}(t)
$ for annuity), and volatility $v^{f}$.

Once the compensation for value transfer occurs, the booking should be
changed (i.e.\ the strike and notional adjusted and the discounting curve
upgraded to the new one), but the volatility surface should not be impacted
by this event.

Cash fee paid on the discounting switch date is a an alternative
compensation option. Under this scenario, a swaption 
\begin{equation*}
V_{\mathrm{swpt}}^{f}(t)=A^{f}(t)\mathrm{E}_{t}^{A^{f}}(S^{f}(T)-K^{f})^{+}
\end{equation*}%
is replaced by a swaption 
\begin{equation*}
V_{\mathrm{swpt}}^{s}(t)=A^{s}(t)\mathrm{E}_{t}^{A^{s}}(S^{s}(T)-K^{s})^{+}
\end{equation*}%
plus a cash payment in the amount of $V_{\mathrm{swpt}}^{f}(t)-V_{\mathrm{%
swpt}}^{s}(t)$. As there is no value transfer, keeping the original
volatilities with the original trade booking and market data
(discounting/projection curves) at the moment of discounting switch
announcement is appropriate. On the actual discounting switch/cash fee
payment event, volatility should not be affected as long as the other
elements are updated, i.e. the fee is paid/received and the discounting
curve is switched to the new one.

\subsubsection{Expectation of Value Transfer}

Now let us consider the scenario where, upon discounting switch
announcement, the market assumes that there will be no compensation for
value transfer. Prices of swaptions will immediately jump (in this idealized
scenario; it will likely take some time for the market to adjust/discover
new prices). In particular, as generally $A^{s}(t)>A^{f}(t)$ (ESTR is lower
than Eonia, while the picture is more mixed with SOFR vs.\ FedFunds), long
swaption positions will generally increase in value.

A \textquotedblleft consensus\textquotedblright\ dealer will switch
discounting to SOFR and keep volatilities unchanged. A \textquotedblleft
contrarian\textquotedblright\ dealer who insists on continuing to use the
legacy FedFunds discounting curve, whether by ignorance, contrarian beliefs,
or system limitations, will have to re-mark volatilities higher. This will
lead to vega PnL (profit/loss arising from volatility changes). While for
the CCP-settled\ swaptions the PnL will be \textquotedblleft
real\textquotedblright\ in the sense of faithfully reflecting market view
for uncompensated value transfer, it may not be so real for other
derivatives that derive their values from swaption volatilities such as
\textquotedblleft legacy\textquotedblright\ pre-November 2018, i.e.\ IRR,
cash-settled European swaptions\footnote{%
In EUR, the IRR vs.\ the \textquotedblleft new\textquotedblright\
cash-settled zero-wide collars is another segment that is currently used to
express views on the probability of compensation for value transfer.} in
EUR, CMS, Bermudan swaptions and other Libor exotics.

The impact of discounting switch announcement on volatilities used by the
contrarian dealer is not the same for different strikes. The value of a
swaption will change as follows (ignoring impact on the forward swap rate
and dropping $t=0$ from notations) 
\begin{equation*}
V_{\mathrm{swpt}}^{f}=A^{f}\mathrm{E}^{A^{f}}(S(T)-K)^{+}\rightarrow V_{%
\mathrm{swpt}}^{s}=A^{s}\mathrm{E}^{A^{s}}(S(T)-K)^{+}.
\end{equation*}%
The new volatility, $v^{s}=v^{s}(K)$ will then be calculated as follows,%
\begin{multline}
v^{s}(K)=C^{-1}\left( \frac{V_{\mathrm{swpt}}^{s}}{A^{f}},K\right)
=C^{-1}\left( \frac{A^{s}}{A^{f}}\mathrm{E}^{A^{f}}(S(T)-K)^{+},K\right)
\label{eq:swpt3} \\
=C^{-1}\left( \frac{A^{s}}{A^{f}}C(v^{f},K),K\right) ,
\end{multline}%
where $C(\sigma ,K)$ is the Normal (or Black) option pricing formula with
volatility $\sigma $ and strike $K$ (with expiry and forward rate dependence
suppressed), and $C^{-1}$ is the inverse of this function in the first,
volatility, argument. It is important to realize that this transformation is
strike-dependent, as signified by the notation $v^{s}=v^{s}(K)$ in (\ref%
{eq:swpt3}). The constant multiplier $V^{s}/V^{f}$ has different impact on
volatilities at different strikes.

Let us present a simple illustration of the impact just described, where for
variety we use representative values from the EUR, rather than USD,
swaptions markets. We consider 10y20y swaptions, i.e.\ swaptions expiring in
10 years into 20 year swaps. Forward swap rate is set to $0.60$\% (60 basis
points, or bps), and we assume that the market has switched to ESTR
discounting while the contrarian dealer still uses Eonia. For clarity we use
the same normal volatility of 50 bps for all strikes as the market. The DV01
ratio in (\ref{eq:swpt3}) is estimated to be $1.0087$, as the relationship
between Eonia and ESTR is already locked down at Eonia = ESTR $+8.5$ bps.

Figure~\ref{fig:swpt_smile_1} shows what volatilities a
legacy(Eonia)-discounting dealer would have to use, compared to the
new(ESTR)-discounting market. The latter are shown as the \textquotedblleft
baseline\textquotedblright\ graph. The former are shown as \textquotedblleft
payers\textquotedblright\ for payer swaption volatilities, and as
\textquotedblleft receivers\textquotedblright\ for receiver swaption
volatilities.

Not only there is a pronounced effect on the volatility smile, the
volatilities implied for payer (call) and receiver (put) swaptions are
worryingly different. This is fairly clear from a brief reflection on (\ref%
{eq:swpt3}) and is unequivocally demonstrated in Figure~\ref%
{fig:swpt_smile_1}. To the extent that high-strike volatilities are
typically implied from the prices of OTM payers and low-strike volatilities
from OTM receivers, this (arbitrage-inducing) mismatch may not be
immediately obvious. Clearly, however, if not recognized, it could lead to
significant mis-pricings in swaption and related markets for the
legacy-discounting dealer.

\begin{figure}[ht]
\includegraphics[origin=c,width=1.0\textwidth]{swpt_smile_impact_01.pdf}
\caption{Impact on implied normal volatilities from mismatched discounting.
``Baseline'' is the baseline volatility. ``Payers'' is the volatility
implied from payer swaptions under mismatched discounting. ``Receivers'' is
the volatility implied from receiver swaptions under mismatched discounting.}
\label{fig:swpt_smile_1}
\end{figure}

\subsubsection{Market Segmentation}

At the time of writing there is still hope that a market-wide solution to
value transfer for swaptions executed before the discount switch
announcement date could be found. As mentioned earlier, however, trading
activity in swaptions (in EUR) post-announcement seems to indicate that at
least some market participants do not believe that this solution, even if
found, will apply to swaptions traded post-announcement and expiring after
the actual discounting switch. If this scenario, i.e.~compensation for
legacy swaptions but not for post-announcement ones, becomes widely accepted
as the most likely outcome, the swaption market will likely segment. Two
swaptions with identical terms (expiry, tenor, strike) traded on different
dates may be valued differently and, in particular, may require different
volatilities and/or different underlying discount curves. At the very least
this development may put yet another strain on internal systems where
potential eligibility of swaptions for compensation, perhaps expressed as
probabilities of various outcomes, need to be captured and used in pricing
and risk management.

The situation is somewhat different in the USD market, at least at the
moment, as there appears to exist a stronger impetus and determination to
find a market-wide solution to the compensation problem, with a consultation
recently launched under the auspices of the Federal Reserve, see \cite{arrc}%
. This market sentiment is supported by a seeming absence of
\textquotedblleft no-compensation\textquotedblright\ swaptions vs.~swaps
trades that we discussed earlier.

\section{Caps}

Interest rate caps present another challenge for the benchmark reform, in
this case caused by term Libor rates being replaced by a daily compounded
in-arrears overnight rate, as mentioned in e.g. \cite{henrard-qse1}, \cite%
{henrard-ssrn1}, \cite{risk-fras}. Here, again, differences between
different currencies emerge as EURIBOR is a go-forward term rate and the
discussion below is only applicable to USD and GBP and, potentially, more so
to GBP, as the US regulators seem somewhat more open to a wider adoption of
alternative \textit{term} rates for derivatives.

Continuing with the notations of Section \ref{sec:swpt_comp}, let us fix $i$
and simplify notations to $L(t)\triangleq L_{i}(t)=L(t,T_{i},T_{i}+\tau ),$ $%
T\triangleq T_{i},$ $\tau \triangleq \tau _{i}$. The rate $L(t)\ $is given by%
\begin{equation*}
L(t)=\frac{P_{i}(t)-P_{i+1}(t)}{\tau P_{i+1}(t)}.
\end{equation*}%
A caplet is an option that pays $\left( L(T)-K\right) ^{+}$ at time $T+\tau $%
, and its value is given by%
\begin{equation}
V_{\mathrm{LibCpl}}(t)=P(t,T+\tau )\mathrm{E}_{t}^{T+\tau }\left(
L(T)-K\right) ^{+}.  \label{eq:cpl1}
\end{equation}

Under the standard fallback mechanism (that is yet to be adopted for
bilateral markets where caps trade, but assuming it will), a Libor rate is
replaced by a compounded in-arrears daily rate with a spread. Under this
scenario the payoff in (\ref{eq:cpl1}) then becomes%
\begin{equation}
V_{\mathrm{OisCpl}}(t)=P(t,T+\tau )\mathrm{E}_{t}^{T+\tau }\left( \frac{1}{%
\tau }\left( \exp \left( \int_{T}^{T+\tau }r(s)~ds\right) -1\right)
-K^{\prime }\right) ^{+},  \label{eq:cpl2}
\end{equation}%
where we use $r(s)$ for the overnight rate, $K^{\prime }$ for the
fallback-spread-adjusted strike, and replace discrete daily with continuous
compounding for notational simplicity.

The fundamental difference between (\ref{eq:cpl1}) and (\ref{eq:cpl2}), as
noted in \cite{henrard-ssrn1}, is that in the former the rate is fixed at
time $T$, whereas in the latter it continues to stochastically evolve until
a later time $T+\tau ,$ combined with the averaging feature of rates fixed
at different times over the time period $[T,T+\tau ].$ A simple European
option in (\ref{eq:cpl1}) becomes an Asian option in (\ref{eq:cpl2}).

Assume for a moment that one-period swaptions are traded. (We will explain
the relevance of this thought experiment in a second.) Its value in the
Libor world is given by%
\begin{equation}
V_{\mathrm{LibSwpt}}(t)=P(t,T+\tau )\mathrm{E}_{t}^{T+\tau }\left( V_{%
\mathrm{LibSwap}}(T)\right) ^{+},  \label{eq:cpl3}
\end{equation}%
where%
\begin{equation}
V_{\mathrm{LibSwap}}(T)=\mathrm{E}_{T}^{T+\tau }\left( L(T)-K\right) =L(T)-K.
\label{eq:cpl4}
\end{equation}%
Clearly (\ref{eq:cpl3})--(\ref{eq:cpl4}) describe a contract equivalent to (%
\ref{eq:cpl1}), and in the Libor world a caplet is equivalent to a
one-period swaption. Now consider the post-Libor situation. Here we have%
\begin{equation}
V_{\mathrm{OisSwpt}}(t)=P(t,T+\tau )\mathrm{E}_{t}^{T+\tau }\left( V_{%
\mathrm{OisSwap}}(T)\right) ^{+}  \label{eq:cpl5}
\end{equation}%
with%
\begin{multline}
V_{\mathrm{OisSwap}}(T)=\mathrm{E}_{T}^{T+\tau }\left( \frac{1}{\tau }\left(
\exp \left( \int_{T}^{T+\tau }r(s)~ds\right) -1\right) -K^{\prime }\right) 
\label{eq:cpl6} \\
\neq \frac{1}{\tau }\left( \exp \left( \int_{T}^{T+\tau }r(s)~ds\right)
-1\right) -K^{\prime }.
\end{multline}%
In fact, combining (\ref{eq:cpl5}) and (\ref{eq:cpl6}) we obtain%
\begin{multline}
V_{\mathrm{OisSwpt}}(t)  \label{eq:cpl7} \\
=P(t,T+\tau )\mathrm{E}_{t}^{T+\tau }\left( \mathrm{E}_{T}^{T+\tau }\left( 
\frac{1}{\tau }\left( \exp \left( \int_{T}^{T+\tau }r(s)~ds\right) -1\right)
\right) -K^{\prime }\right) ^{+}.
\end{multline}%
In contrast to (\ref{eq:cpl2}), part of the expectation value operator moves
inside the $\max (\cdot ,0)$ operator. Jensen's inequality (as also noted in 
\cite{lyas-merc-l}) implies%
\begin{equation}
V_{\mathrm{OisCpl}}(t)\geq V_{\mathrm{OisSwpt}}(t).  \label{eq:cpl8}
\end{equation}

Let us denote (as introduced in \cite{lyas-merc-l}) 
\begin{equation}
R(t)\triangleq R(t,T,T+\tau )=\mathrm{E}_{t}^{T+\tau }\left( \frac{1}{\tau }%
\left( \exp \left( \int_{T}^{T+\tau }r(s)~ds\right) -1\right) \right) ,
\label{eq:termois1}
\end{equation}%
where $t\in \lbrack 0,T+\tau ]$, i.e.~$t$ is allowed to be larger than $T$
(as in \cite{lyas-merc-l}). It is trivially a martingale in $\mathrm{P}%
^{T+\tau }$ so can easily be used as the underlying in the standard
\textquotedblleft vanilla\textquotedblright , such as Black-Scholes or SABR,
modelling.  For $t\leq T$ this is called the (forward) term one-period
compounded OIS swap rate and is defined as the break-even rate on a
one-period forward starting OIS swap observed at time $t$. With this
notation we have%
\begin{equation*}
_{\mathrm{OisSwap}}(T)=\mathrm{E}_{T}^{T+\tau }\left( R(T)-K^{\prime
}\right) 
\end{equation*}%
which is the equivalent of (\ref{eq:cpl4}) in post-Libor world. We note that
the caplet value (\ref{eq:cpl2}) is then%
\begin{equation*}
V_{\mathrm{OisCpl}}(t)=P(t,T+\tau )\mathrm{E}_{t}^{T+\tau }\left( R(T+\tau
)-K^{\prime }\right) ^{+}.
\end{equation*}

\subsection{Market Impact}

There are a number of implications of the differences just explained that
are worth exploring. Whereas a standard interest rate swap post Libor
transition looks essentially like a swap before transition as far as
interest rate risk is concerned, caps change their character significantly,
and may no longer be suitable for market participants who have them on the
books as, for example, a loan hedge. Sensitivity to volatility looks
different, and the complexity of valuation is different as well, potentially
affecting liquidity and the ability to exit positions.

There is also, of course, the issue of value transfer. Before the talk of
Libor transition began, one could enter into a long Libor caplet hedged by a
short matching single period swaption at zero cost. While single-period
swaptions are not really traded, a cap versus a matching tenor swaption, for
example a one-year forward starting cap on three-month Libor versus a
one-year swaption with a matching first expiry, was at times a popular hedge
fund trade. Originally designed as a bet on Libor/swap rate correlations,
under the fallback protocol not only will it likely change its value, but
will also acquire different risk characteristics.

\cite{henrard-blog1} proposes a different fallback mechanism for caps that
would keep them aligned with single-period swaptions -- we discuss it in
more detail later in the context of other products.

A similar divergence may occur between caplets and exchange-traded options
on Eurodollar futures.

\subsection{Volatility Adjustment}

Let us estimate the difference in value of a caplet vs.~a matching
single-period swaption in the post-Libor world. For a very simple estimate,
let us assume that $r(t)$ follows a Brownian motion with constant volatility 
$\sigma $, and let us approximate%
\begin{equation}
R(T+\tau )=\frac{1}{\tau }\left( \exp \left( \int_{T}^{T+\tau
}r(s)~ds\right) -1\right) \approx \frac{1}{\tau }\int_{T}^{T+\tau }r(s)~ds.
\label{eq:va1}
\end{equation}%
To estimate the variance of $R(T+\tau ),$ we recall (see Appendix) that for
the standard Brownian motion $W(\cdot )$ for $T\geq 0$%
\begin{equation*}
\mathrm{E}\left( \left( \int_{T}^{T+\tau }W(s)~ds\right) ^{2}\right) =\tau
^{2}\left( T+\frac{\tau }{3}\right) .
\end{equation*}%
Thus%
\begin{equation*}
\mathrm{Var}\left( R(T+\tau )\right) =\sigma ^{2}\left( T+\frac{\tau }{3}%
\right) .
\end{equation*}%
At the same time 
\begin{equation*}
R(T)=r(T)
\end{equation*}%
under the Brownian motion assumption and the approximation (\ref{eq:va1}),
so that%
\begin{equation*}
\mathrm{Var}\left( R(T)\right) =\sigma ^{2}T.
\end{equation*}%
The ratio of volatilities is then given by (see also \cite{lyas-merk-ssrn})%
\begin{equation}
\frac{\mathrm{Vol}\left( R(T+\tau )\right) }{\mathrm{Vol}\left( R(T)\right) }%
=\sqrt{1+\frac{\tau }{3T}}.  \label{eq:cpl9}
\end{equation}%
It is always larger than $1$ as already discussed, and could be
significantly larger than $1$ for shorter-dated options i.e.~smaller $T.$

\subsection{Considerations for Modelling}

Modeling caps post-Libor transition is more complicated as the option
becomes Asian. In this regard, \cite{henrard-ssrn1} derives valuation
formulas in a one-factor Gaussian model (see also \cite{henrard-cr1}, \cite%
{henrard-cr2}), and of course the seminal extension of the Libor market
model for overnight rates from \cite{lyas-merc-l} can be used. Both,
however, are likely to be deemed impractical by traders for what would still
be likely considered a vanilla product. The Gaussian model is likely to be
ruled out on the basis of not supporting the smile natively, and the LMM
because of computation cost (and also difficulties in controlling the
smile). Hence, an adaptation of a vanilla model such as SABR will likely be
required.

Complicating matters somewhat is the uncertainty that surrounds fallback
mechanisms as applied to caps. As we have seen, a straightforward
application of the standard fallback makes caps materially different from
what they were/are now. It is possible that the industry will converge on a
different solution for caps, e.g. replacing them with (strips of) single
period swaptions -- essentially creating instruments linked to term
risk-free rates (as swap rates on single-period OIS swaps essentially are),
without calling them as such.

While the future is uncertain, it is possible that both types of contracts,
Asian-style OIS caplets and European style single-period swaptions, will be
traded in the future. This possibility should be one of the considerations
of any future vanilla risk management framework. In particular the model
should be flexible enough so that the two types of contracts can be marked
at different levels of volatility and, potentially, of other volatility
smile parameters. In particular, it seems unlikely that a simple scaling of
swaption volatility along the lines of (\ref{eq:cpl9}) would be sufficient
to match both markets.

The relation (\ref{eq:cpl9}) does, however, have a purpose as it expresses
volatilities of different contracts in a comparable way. Let us assume we
mark single-period OIS swaptions with expiry $T$ and tenor $\tau $ with
parameters (in the standard SABR parameterization) $\sigma _{\mathrm{swpt}},$
$\alpha _{\mathrm{swpt}},$ $\beta _{\mathrm{swpt}},$ $\rho _{\mathrm{swpt}}$%
. It is natural then to use a SABR model for caplets as well. We would then
mark $\sigma _{\mathrm{cpl}}$ as we see fit and, perhaps, in the first
iteration of the model use the same smile parameters 
\begin{equation*}
\alpha _{\mathrm{cpl}}=\alpha _{\mathrm{swpt}},\quad \beta _{\mathrm{cpl}%
}=\beta _{\mathrm{swpt}},\quad \rho _{\mathrm{cpl}}=\rho _{\mathrm{swpt}}.
\end{equation*}%
Before applying the SABR model to a caplet we would first transform $\sigma
_{\mathrm{cpl}}$ into an approximation to the \textquotedblleft Asian
volatility\textquotedblright\ using (\ref{eq:cpl9}) by setting%
\begin{equation}
\sigma _{\mathrm{asian}}=\sigma _{\mathrm{cpl}}\sqrt{1+\frac{1}{3}\frac{\tau 
}{T}}  \label{eq:cpl10}
\end{equation}%
for $T\geq 0$ (the case $-\tau <T<0$ should obviously be treated somewhat
differently -- we omit obvious details). Finally, we would use the set of
parameters 
\begin{equation*}
\sigma _{\mathrm{asian}},\alpha _{\mathrm{cpl}},\beta _{\mathrm{cpl}},\rho _{%
\mathrm{cpl}}
\end{equation*}%
in the SABR formula to calculate (\ref{eq:cpl2}) with the strike $K^{\prime }
$ and the forward set to 
\begin{equation*}
R(0)=\mathrm{E}^{T+\tau }\left( \frac{1}{\tau }\left( \exp \left(
\int_{T}^{T+\tau }r(s)~ds\right) -1\right) \right) 
\end{equation*}%
(or the exact daily compounding version thereof). We remind the reader that
while this expression seems to indicate that some convexity corrections are
required because of the difference of the payment and fixing times, it is in
fact these quantities that are bootstrapped from the (eventually liquidly
traded) compounded OIS in-arrears swaps, so in fact are a direct market
input.

The scaling (\ref{eq:cpl10}), while not strictly necessary, allows traders
to think of caplets versus swaptions, and caplets of different expiries, in
normalized units, as they do not need to worry about the translation of
caplet volatilities to the same units as swaptions volatilities and the
like. We have not performed a similar re-scaling for other SABR parameters
as we consider their impact to be of second-order that will possibly come
into effect later as the markets develop (if, in fact, they develop as we
envisage). When we do have liquid caplet and single-period swaption markets,
the question of how to adjust other SABR parameters for Asian features may
need to be tackled.

\section{Disappearing Species?}

Interest rate caps change quite significantly under the standard Libor
fallback. Some of the products may be affected even more and disappear
altogether. Let us look at some examples.

\subsection{Libor-In-Arrears\label{sec:lia}}

In a Libor-In-Arrears (LIA, see \cite{ap-book}) swap, the floating leg pays
a Libor rate as soon as it is fixed, unlike a Libor rate in the standard
swap paid at the end of the accrual period. The fallback for Libor, the
in-arrears compounded overnight rate, works well for a standard swap but not
for an in-arrears swap. The Libor replacement rate will simply not be known
at the beginning of the accrual period. So while caps get transformed into
meaningful, although different from the original, contracts, LIA swaps
simply do not work under the standard fallback. This is similar to the
situation with FRAs as discussed in \cite{henrard-qse1}. Legal recourse of
counterparties under a contract that cannot be fulfilled due to time being
unidirectional is a fascinating topic that space limitations prevent us from
exploring. Instead we look at how this dilemma could be prevented by
modifying LIA swaps accordingly. There are two angles that we can think of,
and we consider them in turn.

Clearly a term version of the overnight rate -- a fixed break-even rate one
would pay on a one-period swap against the compounded in-arrears overnight
rate such as (\ref{eq:termois1}) -- would be a near perfect substitute for a
Libor rate in an LIA swap, as well as for many other Libor-linked contracts
such as FRAs. Regulators so far have discouraged reliance on the potential
existence of such rates for derivatives fallback. It can, however, be
created synthetically. A practical suggestion in this spirit is proposed in 
\cite{henrard-blog1}. Alternatively, before Libor disappears, an LIA swap
can be restructured into a contract that, on each fixing date, observes a
prevailing swap rate on a single-period swap vs.~the daily compounded rate,
e.g.~the upcoming ICE RFR swap rate\footnote{%
The so-called screen swap rate produced by ICE from cleared (Libor at the
moment) swap quotes and currently used for settling swaptions in EUR. ICE is
reportedly planning to introduce RFR-based swap rates in April 2020. } on
the shortest supported tenor, and pays an amount linked to that. It is not
however clear that such a rate would be reliably observable, or regulators
would not actively discourage contracts with de-facto links to term rates.

Let us now consider what can be done with overnight compounded rates. The
prevailing commercial rationale for customers to trade LIA, rather than
standard, swaps is that on a receiver swap (i.e.\ the client receives fixed,
pays Libor), the client would pay Libor-based payments sooner than in a
normal swap and the dealer may offer a (marginally) higher fixed rate for
that. Moreover, dealers receiving Libor in arrears benefit from convexity
(see \cite{ap-book}) and may share some of that by paying an even higher
fixed rate to the client.

Let us briefly recall LIA convexity. Libor rate $L(t)=L(t,T,T+\tau ),$ if
paid in-arrears, is worth at time $t$ (see \cite{ap-book})%
\begin{equation*}
V_{\mathrm{LibLis}}(t)=P(0,T)\mathrm{E}_{t}^{T}L(T)=P(0,T+\tau )\mathrm{E}%
_{t}^{T+\tau }\left( L(T)\left( 1+\tau L(T)\right) \right) .
\end{equation*}%
A payment of Libor at time $T$ is financially equivalent to a payment of $%
L(T)\left( 1+\tau L(T)\right) $ at time $T+\tau .$ The latter is much more
amendable to the proposed standard fallback, and we can simply replace an
LIA swap with a contract that pays%
\begin{equation*}
\tilde{R}\left( 1+\tau \tilde{R}\right)
\end{equation*}%
at time $T+\tau ,$ where $\tilde{R}=R(T+\tau )+F$ as defined by (\ref%
{eq:termois1}), with $F$ being the fallback spread. Note that a similar fix
works for an FRA.

\subsection{Range Accruals}

Things get more complicated still with range accruals, a popular Libor
exotic often embedded in structured notes, sometimes in callable form (see 
\cite{ap-book}). A basic range accrual (RA) coupon pays (a daily discretized
version of)%
\begin{equation}
\int_{T}^{T+\delta }1_{\{L_{0}(t)>a\}}dt  \label{eq:RA1}
\end{equation}%
at time $T+\delta ,$ where $L_{0}(t)\triangleq L(t,t,t+\tau )$ is the
current time-$t$ (signified by the subscript $0$) Libor fixing, and $a$ is a
lower barrier. We note that many other versions of range accruals exist, see 
\cite{ap-book}.

As with LIA, a direct replacement of $L_{0}(t)$ with the standard fallback
compounded overnight rate does not work as it is not known at time $t$, and
hence the payment cannot be made at $T+\delta .$ There are also more subtle
issues that we will come to in a moment.

Clearly, the RA contract would also benefit from the existence of term
risk-free rates. In the absence of the official term fixings, they could
possibly be simulated by referencing the breakeven fixed rate on a
one-period swap vs.~compounded daily rate such as the ICE rate as mentioned
in Section~\ref{sec:lia}.

Let us see what happens if we directly replace $L_{0}(t)$ in (\ref{eq:RA1})
with $R_{0}(t)\triangleq R(t+\tau ,t,t+\tau ),$ where the subscript $0$
signifies that the rate starts compounding immediately on the observation
date. Ignoring the adjustment of the barrier $a$ for the Libor-OIS fallback
spread, the payoff becomes%
\begin{equation*}
\int_{T}^{T+\delta }1_{\{R_{0}(t)>a\}}dt,
\end{equation*}%
and is fully known only at time $T+\delta +\tau ,$ and not at time $T+\delta 
$ for the original contract. Therefore it cannot be paid earlier than $%
T+\delta +\tau $; to preserve the economics of discounting we can compound
it by the rate from $T+\delta $ to $T+\delta +\tau $, thus replacing (\ref%
{eq:RA1}) with%
\begin{equation}
\left( 1+\tau R_{0}\left( T+\delta \right) \right) \int_{T}^{T+\delta
}1_{\{R_{0}(t)>a\}}dt  \label{eq:RA2}
\end{equation}%
paid at $T+\delta +\tau .$

A note of caution is in order here. While paying a certain amount on a given
date is economically equivalent to paying a properly un-discounted value at
a later date, operationally these two payments are not the same. For
example, an investor in a structured note with an embedded range accrual
rightfully expects coupon and principal payments on the agreed-upon dates
and not on some future dates, and may be reluctant to accept changes. This
is yet another manifestation of the practical challenges of the benchmark
reform implementation for structured products.

This is not the end of the story, however. The same issue that we had with
Libor caps turning \textquotedblleft Asian\textquotedblright\ under the
fallback is present here as well. Whereas an option that pays $%
1_{\{L_{0}(t)>a\}}$ is a European-style digital option and can be valued
directly from the caplet smile for time $t,$ an option that pays $%
1_{\{R_{0}(t)>a\}}$ is an option on an average rate with the observations
for the average covering the time interval $[t,t+\tau ].$ The underlying
rate will have higher volatility but the impact of that on the digital
option value can be in either direction. Hence, even if we go into the
trouble of restructuring range accrual swaps with clients from paying (\ref%
{eq:RA1}) to (\ref{eq:RA2}), there will still be potentially significant
value transfer and risk impact that would need to be dealt with.

\section{Conclusions}

We explored a number of areas where rates benchmark reform affects
non-linear rates markets in significant ways. While by no means exhaustive,
even the cases considered demonstrate the sheer amount of further
considerations and work required to bring Libor termination to a successful
conclusion. We have made a number of practical suggestions on how the
biggest challenges could be overcome. It is clear however that replacing
term Libor rates with compounded overnight rates in all possible use cases
in non-linear markets remains a formidable challenge.

\section*{Acknowledgments}

I would like to thank Pierre-Yves Guerber, Phil Lloyd, Oliver Cooke,
Vladimir Golovanov, Oscar Arias, Marc Henrard and Andrei Lyashenko for
thoughtful comments and discussions. All remaining errors are mine.

\section*{Disclaimer}

Opinions expressed in this article are those of its author and do not
necessarily reflect the views and policies of NatWest Markets or any other
organization.

\begin{thebibliography}{99}
\bibitem{ap-book} L.~Andersen and V.~Piterbarg, "`Interest Rate Modelling",
in three volumes, 2010, Atlantic Financial Press

\bibitem{arrc} Alternative Reference Rates Committee, ``ARRC Consultation on
Swaptions Impacted by the CCP Discounting Transition to SOFR'', February
2020, https://www.newyorkfed.org/medialibrary/Microsites/arrc/files/ 2020/
ARRC\_Swaption\_Consultation.pdf

\bibitem{risk-presess} H. Bartholomew, "LCH targets hardwired pre-cessation
triggers", January 2020, Risk.net

\bibitem{risk-bonds} B. St. Clair. "Libor fallbacks a low priority for most
bond investors", March 2019, Risk.net

\bibitem{cme-ds} CME group, ``SOFR Discounting \& Price Alignment Transition
Plan for Cleared USD Interest Rate Swaps'', December 2019,
www.cmegroup.com/education/articles-and-reports/sofr-price-alignment-and-discounting-proposal.html

\bibitem{risk-mh1} M. Henrard, "Signing the Libor fallback protocol: a
cautionary tale", January 2020, Risk.net

\bibitem{henrard-qse1} M. Henrard, "A Quant Perspective on LIBOR fallback",
March 2019, Quant Summit Europe conference

\bibitem{henrard-ssrn1} M. Henrard, "A Quant Perspective on IBOR Fallback
Consultation Results - V2.1", January 2019, Available at SSRN:
https://ssrn.com/abstract=3308766

\bibitem{henrard-blog1} M. Henrard, "LIBOR Fallback: is physical settlement
an alternative? - Financial fiction", July 2019,
http://multi-curve-framework.blogspot.com/2019/07/libor-fallback-is-physical-settlement.html

\bibitem{henrard-cr1} M. Henrard, "Overnight Indexed Swaps and Floored
Compounded Instrument in HJM one-factor model", Februrary 2008. Economics
Working Paper Archive, https://ideas.repec.org/p/wpa/wuwpfi/0402008.html.

\bibitem{henrard-cr2} M. Henrard, "Skewed Libor Market Model and Gaussian
\{HJM\} explicit approaches to rolled deposit options", The Journal of Risk,
9(4)

\bibitem{lyas-merc-l} A. Lyashenko and F. Mercurio, "Libor replacement: a
modelling framework for in-arrears term rates", July 2019, Risk Magazine

\bibitem{lyas-merk-ssrn} A. Lyashenko and F. Mercurio, "Looking Forward to
Backward-Looking Rates: A Modeling Framework for Term Rates Replacing
LIBOR", February, 2019, Available at SSRN: https://ssrn.com/abstract=3330240

\bibitem{risk-swpt} R. Mackenzie Smith, "LCH won't back single fix for
swaptions", November 2020, Risk.net

\bibitem{vp-disc} V. Piterbarg, "Funding beyond discounting: collateral
agreements and derivatives pricing", February 2010, Risk Magazine, pp.97--102

\bibitem{risk-fras} N. Sherif, "FRAs won't work with standard Libor
fallback, experts say", March 2019, Risk.net
\end{thebibliography}

\newpage

\appendix{}

\section{Appendix}

For $T\ge 0$, 
\begin{align*}
\mathrm{E}\left( \left( \int_{T}^{T+\tau }W(s)~ds\right) ^{2}\right) &
=\int_{T}^{T+\tau }ds\int_{T}^{T+\tau }du~\mathrm{E}\left( W(s)W(u)\right) \\
& =\int_{T}^{T+\tau }ds\int_{T}^{T+\tau }du~s\wedge u \\
& =2\int_{T}^{T+\tau }ds\int_{s}^{T+\tau }s~du \\
& =2\int_{T}^{T+\tau }s(T+\tau -s)ds \\
& =\tau ^{2}\left( T+\frac{\tau }{3}\right) .
\end{align*}

The formula is easily generalized for all possible values of $T$, i.e.\ for
all $T\ge -\tau$:

\begin{equation*}
\mathrm{E}\left( \left( \int_{T}^{T+\tau }W(s)~ds\right) ^{2}\right) = \frac{%
1}{3}\left(T-\tau\right)^3 - \left(T^+\right)^2\left( T+\tau - \frac{2}{3}
T^+ \right),
\end{equation*}
where $T^+= \max(T,0)$.

\end{document}
