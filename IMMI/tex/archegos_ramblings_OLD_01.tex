
\documentclass{article}
\usepackage{amsmath}

%%%%%%%%%%%%%%%%%%%%%%%%%%%%%%%%%%%%%%%%%%%%%%%%%%%%%%%%%%%%%%%%%%%%%%%%%%%%%%%%%%%%%%%%%%%%%%%%%%%%%%%%%%%%%%%%%%%%%%%%%%%%%%%%%%%%%%%%%%%%%%%%%%%%%%%%%%%%%%%%%%%%%%%%%%%%%%%%%%%%%%%%%%%%%%%%%%%%%%%%%%%%%%%%%%%%%%%%%%%%%%%%%%%%
%TCIDATA{OutputFilter=LATEX.DLL}
%TCIDATA{Version=5.50.0.2953}
%TCIDATA{<META NAME="SaveForMode" CONTENT="1">}
%TCIDATA{BibliographyScheme=Manual}
%TCIDATA{Created=Saturday, June 05, 2021 09:00:47}
%TCIDATA{LastRevised=Monday, June 07, 2021 12:34:12}
%TCIDATA{<META NAME="GraphicsSave" CONTENT="32">}
%TCIDATA{<META NAME="DocumentShell" CONTENT="Standard LaTeX\Blank - Standard LaTeX Article">}
%TCIDATA{CSTFile=40 LaTeX article.cst}

\newtheorem{theorem}{Theorem}
\newtheorem{acknowledgement}[theorem]{Acknowledgement}
\newtheorem{algorithm}[theorem]{Algorithm}
\newtheorem{axiom}[theorem]{Axiom}
\newtheorem{case}[theorem]{Case}
\newtheorem{claim}[theorem]{Claim}
\newtheorem{conclusion}[theorem]{Conclusion}
\newtheorem{condition}[theorem]{Condition}
\newtheorem{conjecture}[theorem]{Conjecture}
\newtheorem{corollary}[theorem]{Corollary}
\newtheorem{criterion}[theorem]{Criterion}
\newtheorem{definition}[theorem]{Definition}
\newtheorem{example}[theorem]{Example}
\newtheorem{exercise}[theorem]{Exercise}
\newtheorem{lemma}[theorem]{Lemma}
\newtheorem{notation}[theorem]{Notation}
\newtheorem{problem}[theorem]{Problem}
\newtheorem{proposition}[theorem]{Proposition}
\newtheorem{remark}[theorem]{Remark}
\newtheorem{solution}[theorem]{Solution}
\newtheorem{summary}[theorem]{Summary}
\newenvironment{proof}[1][Proof]{\noindent\textbf{#1.} }{\ \rule{0.5em}{0.5em}}
\input{tcilatex}

\begin{document}


\section{Background}

Consider a hedge fund (HF) that wants to enter into a position referencing a
risk factor $S=S(t)$ that, for simplicity, we call a stock. HF can enter
into a cash position with its prime broker (PB) by

\begin{itemize}
\item Borrowing money from PB

\item Buying the stock

\item Pledging the stock as collateral for the loan
\end{itemize}

A PB would require a haircut, i.e. it would lend only, say, 90\% of the
values of the shares it holds as collateral. So the economics, for a 100
unit transaction, would be

\begin{itemize}
\item HF borrows enough money from PB\ for 90 units

\item HF pays for 10 units from its own funds

\item HF gives 100 units to the PB as collateral
\end{itemize}

In this case, if the value of the stock declines, HF\ needs to post more
collateral (cash or other assets) to PB to make sure the value of the
collateral the PB holds is at least 100/90 of the value of the initial loan.
The "haircut" of 10 units is there is for PB's protection in case of the HF\
not being able to post more collateral when required ("margin call"). The
theory is then that if HF misses a margin call, the PB can sell the shares
over a period of time (an input to the margin model, often 10 days or so) to
recover its full loan value to PB even if shares are volatile over the
period of liquidation (MPOR).

Regulators require disclosure of large stock holdings so the whole market
can see them. Some HFs may consider it advantageous to enter into positions
that are not subject to disclosure. To that effect, a HF can enter, say, a
total return swap on a particular stock, a derivative transaction with the
PB\ that pays to the HF\ returns of the stock but, critically, is not
subject to the same level of disclosure as physical equity holdings.
Likewise, they can enter into other derivative transactions such as calls or
collars (or puts, if they are betting against the stock).

The mechanics of a total return swap are designed to closely replicate that
of physical equity transaction but there some differences. A HF would agree
to pay a certain rate (Libor or RFR+spread) and receive changes in MTM on
the stock. The PB would buy the stock and hold it on its own books to hedge
its obligations to the HF. The PB would borrow money from, say the treasury
desk to buy the initial stock. While HF is solvent, PB\ is "riskless" in the
sense that price changes on its holdings of the stock are just passed on to
the client who bears market risk. If HF defaults for whatever reason, the PB
will naturally stop paying/passing on stock returns to the HF and will
suddenly be left with an unhedged position in the stock. Naturally, it will
try to unwind it by selling its heding position. If the movement in the
stock made it so that the value is less than what the PB owes to the
treasury for the initial purchase, it would have to realize a loss. 

Hence, the PB would typically require margins from PB. There are generally
two kinds. One is the variation margin which ensures that the PB's stock
value+variation margin is (roughly) equal to the original price the PB paid
to buy the stock (so what it borrowed from the treasury). It can also be
seen as the MTM on the TRS vs the HF, so essentially a standard CSA. This is
a synthetic equivalent of the margin calls a HF would get on a physical
stock holding. 

The PB would also require the initial margin that would allow it to unwind
its stock position in orderly fasion over a number of days, should the HF
default and the stock move adversely over that many days. This the is
equivalent of the "haircut" in the physical equity holding scenario.

\section{Setup}

Let $S=S(t)$ be the stock. Let us assume the underlying annualized lognormal
volatility of the stock is $\sigma $. Consider $N+1$ primer brokers. We
consider the situation from the point of view of broker $0.$ Let us assume
that the HF has a total return swap position in $S$ against broker $0$ We
denote the position that the PB $0$ has against HF as $\pi _{0}.$ This means
that, if fully hedged, the PB $0$ paid%
\[
-\pi _{0}S(0)
\]%
at the time of the TRS initiation, and the value of its hedge at time $t$ is
given by%
\[
\pi _{0}S(t).
\]%
So if the HF is long $S,$ the PB's hedging portfolio is also long $S$ and
the MTM on the hedge portfolio for PB is%
\[
\pi _{0}\left( S(t)-S(0)\right) .
\]%
To be clear, if the HF is long $S,$ then PB is short $S$ on TRS, but long $S$
in the hedge so $\pi _{0}>0.$

\section{Stylized margin model}

A broker requires an initial margin and a variation margin to be posted by
the HF to cover its exposure to the risk factor $S.$ Simplistically, let us
assume that the margin model is specific to exposure $S$ (no diversification
benefits etc). The broker $0$ would require a variation margin $%
V_{0}=V_{0}(t)$ that, when added to its hedge portfolio MTM, makes it
roughly flat:%
\begin{equation}
V_{0}(t)+\pi _{0}\left( S(t)-S(0)\right) =0.  \label{eq:vm1}
\end{equation}
(XXX: does PB post VM to HF if in their favour? I think so but worth
checking I suppose). In particular this states that the daily change in the
variation margin is given by%
\[
V_{0}(t+dt)-V_{0}(t)=-\pi _{0}\left( S(t+dt)-S(t)\right) ,
\]%
where we used $dt$ to denote one day, $dt=1/250.$ We see that the variation
margin is the change in the MTM for PB on the TRS vs HF (as $-\pi _{0}$ is
the exposure of the PB on the TRS), just like we would expect in the normal
CSA setup, see [VP1].

In addition to the variation margin, PB requires an initial margin $%
I_{0}=I_{0}(t)$ posted, whose purpose is to cover broker's costs if he needs
to unwind the position should the client default, over an MPOR of $\mu _{0}$
(say 10 days or so, $\mu _{0}=10/250$). A very stylized margin model would
specify the initial margin to be a high percentile, $\alpha $ (say, $\alpha
=99\%$) of the lower tail of its exposure to $S$ over the period of $%
[t,t+\tau _{0}],$ assuming immediate default of HF at $t,$ implicitly
defined by%
\[
\mathrm{P}\left( V_{0}(t)+\pi _{0}\left( S(t)-S(0)\right) +\pi _{0}\left(
S(t+\mu _{0})-S(t)\right) +I_{0}(t)>0\right) =\alpha .
\]%
The value of the assets held by the PB in this expression are

\begin{itemize}
\item The VM

\item The value of the stock hedge minus the price that was paid for it

\item The (stochastic)\ change in the value of the stock hednge over $\mu
_{0}$

\item The initial margin that cover possible loss on this (being $<0$) to $%
1-\alpha $ probability
\end{itemize}

We note that under the assumption (\ref{eq:vm1}) this simplifies to%
\[
\mathrm{P}\left( \pi _{0}\left( S(t+\mu _{0})-S(t)\right) +I_{0}(t)>0\right)
=\alpha .
\]%
We however need to relax the assumption of the so-called
"zero-threshold"/daily margin posting (\ref{eq:vm1}) to make the model
realistic enough to describe the effects that we are after. We explain this
next.

\section{Thresholds}

Variation margin may not be collected daily. Some margin agreements
stipulate that the underlying position needs to move by a sufficient amount
for the margin call to be made, the so-called \emph{threshold} (XXX:\ is
this the right term). There could be two reasons for this. One is
operational -- posting margin (essentially sending cash by wire transfer)
costs a fixed amount of money both in fees and operational costs for both
parties, and the agreement could be made that only when sufficiently
significant change in required margin accumulates, the payment is made. 

Another reason why we would want to introduce thresholds into the model we
are building is behavioral. For a large position with, presumably, an
important client, a PB would have a certain lag between a margin call being
missed and the decision to start unwinding the underlying exposure. Lag
times are institution-specific. To keep things simple, we roll this into the
"implied" threshold that we bundle up with any actual thresholds that are
hard-wired into the PB agreement (XXX: is this reasonable? do we want to
model decision lag times explicitly as times not as margin levels?)

It is not uncommon (XXX: is it?) to have the threshold to be at or around
the level of the initial margin. The idea being that as long as the initial
margin plus already-posted variational margin covers the current position,
it is fine, and only needs to be topped up once this level is breached. This
of course does not make a lot of sense from the definitions of the
variational margin (current MTM) and the initial margin (protection against
future changes in MTM\ in the case of default) but practicalities or perhaps
"market practice" can lead to this (XXX This paragraph is perhaps irrelevant
and we should keep IM and thresholds separate. Poor risk management implies
lower IM and higher threshold, either real or implied, so the two are
actually anti-correlated in a sense). 

Let the threshold (a combination of real and implied) be $R_{n},$ $n=0,\dots
,N.$

It at time $t$ the value of the hedge position is $\pi _{0}\left(
S(t)-S(0)\right) $ and the value of the VM is $V_{0}(t),$ then the next
margin call for PB\ $0$ is given by the the (stochastic) time $\rho _{0}$
such that%
\[
\rho _{0}=\inf_{u\geq t}\left\{ \pi _{0}\left( S(u)-S(0)\right)
+V_{0}(t)=-R_{0}\right\} 
\]%
and then, assuming normal operations, 
\begin{align*}
V_{0}(\rho _{0})& =-\pi _{0}\left( S(\rho _{0})-S(0)\right) , \\
V_{0}(\rho _{0})-V_{0}(t)& =-\pi _{0}\left( S(\rho _{0})-S(0)\right)
-V_{0}(t)=R_{0}.
\end{align*}

If the HF defaults on the margin call at this point, the broker has

\begin{itemize}
\item VM of $V_{0}(t)$

\item A hedge position that is worth $\pi _{0}\left( S(\rho
_{0})-S(0)\right) $ (this includes the need to repay the original loan from
the treasury to buy the stock)

\item The IM $I_{0}(t)$ (XXX here we assume the IM\ and VM are updated at
the same time following the threshold concept we introduced earlier)
\end{itemize}

The sum of the first two items is the negative of the threshold $-R_{0}$. So
there is can in the amount of $I_{0}(t)-R_{0}$ and $\pi _{0}$ worth of stock 
$S$ to unwind. We assume the broker start selling immediately. 

\section{Outline of the model}

We have $N+1$ brokers. We assume time now is $t_{0}>0.$ PBs have exposures $%
\pi _{n}$ to $S,$ hold $V_{n}(t_{0})$ VMs and $I_{n}(t_{0})$ IMs. They have
their thresholds $R_{n}$ and "next" margin calls of $\rho _{n}.$ Here $%
n=0,\dots ,N+1.$ 

The "normal" evolution of the stock is given by 
\[
dS(t)/S(t)=\sigma ~dW(t)
\]%
where we assumed no drift (in the physical measure) because of relatively
short time scales involved. 

The model we are trying to build can be described as follows

\begin{enumerate}
\item $S(t)$ goes down for whatever reason (we are assuming here HF\ is long
vs all the PBs)

\item The PB with the shortest $\rho _{n}$ (smallest threshold $R_{n}$
versus the gap between their VM and the value of their hedge portfolio)
makes a margin call. Let's call this PB $\eta _{0}$

\item The HF does not make the margin call

\item PB $\eta _{0}$ starts selling

\item This moves the stock lower -- here we use some sort of market impact
model

\item This triggers the margin call on the next PB ($\eta _{1}$). They start
selling, increasing the downward drift

\item This increases the selling pressure and triggers PB $\eta _{2}$ etc

\item Everyone loses because of the strong downward drift not acccounted for
in IM (not to mention delayed response time via $R_{n}$). But not the same
amount -- the one with the tightest threshold $R_{n}$ and highest IM $I_{n}$
loses the least
\end{enumerate}

The goal is to determine the distribution of losses for PB $0$ ("us") given
that we do not have exact information about the positions and margins and
margin agreements that all the other PBs $n=1,\dots ,N$ have but can put
some distributions around them. Once the distribution of losses is
estimated, it should be capitalized and/or extra "concentration" margin
charged to the client as a mitigant.

\section{Further on thresholds}

Come to think of it, we should, after all, probably separate the real
thresholds from the delays in execution. So let us assume that the real
threshold is $R_{n}$, but once the margin call is missed, we assume PB $n$
waits for $\omega _{n}$ time before starting to execute its unwinds. Here $%
\omega _{n}$ is a random time, and we can recover our original setup by
setting it to the first hitting time of the "implied" threshold. But for
greater generality, and to study the effects of delays in execution
separately from the markgin agreement details, we can keep it separate.

\section{Market impact}

The main point of our model is that the stock drifts lower and lower as more
and more PBs starts selling their hedge portfolios. A standard model of
market impact is the so-called square root law, see [Gatheral slides], [Toth
et all],
[https://quant.stackexchange.com/questions/41937/market-impact-why-square-root], [Kyle Obizhaeva] (XXX need a better reference here?) It states that the stock, apart from volatility, experiences price changes due to selling, in the form of%
\[
\frac{S(t_{k})-S(t_{k}-)}{S(t_{k}-)}\sim \limfunc{sign}(q_{k})\times c\times 
\hat{\sigma}\sqrt{\frac{\left\vert q_{k}\right\vert }{\nu }}
\]%
where $q_{k}$ is the number of shares transacted at $t_{k}$, $\nu $ is the
daily volume in shares, $\hat{\sigma}$ is the \textbf{daily} volatility
unlike before and $c$ some constant empirically observed to be close to $1.$ 

\begin{remark}
We worry about the impact of LARGE trades, does that mean we need to look at
different kinds of results? [Toth] states p2: "But as T increases, such as
the number of trades becomes large, the relation between T and QT becomes
more and more linear for small imbalances (see e.g.[3], Fig. 2.5)...", but
also adds "...except in rare cases when $Q(T)/V$ is large, in any case much
larger than the region where Eq. (1) holds". Goes on to duscuss square root
singularity for small trades. However [Bershova et al] look specifically at
large orders at also seems to see the square root there, although I did not
read the paper carefully yet. From abstract "We examine price reversion
after the completion of a trade, finding that permanent impact is also a
square root function of trade duration", however I am not sure what "trade
duration" is here
\end{remark}

\section{The Model}

Let us try to formulate the model mathematically. The notations are going to
be clunky. 

The standard IM\ model calculates possible loss on the market position under
the "orderly" unwind given an instantaneous default of the HF at time $T_{0}.
$

We would like co calculate the adjustment to that given "disorderly" unwind.

At time $T_{0},$ each PB $n$ holds a cash position%
\[
C_{n}(T_{0})=-\pi _{n}S(0)+V_{n}(T_{0})+I_{n}(T_{0})
\]%
and a stock position%
\[
V_{n}(T_{0})=\pi _{n}S(T_{0})
\]%
We note that 
\[
\pi _{n}\left( S(T_{0})-S(0)\right) +V_{n}(T_{0})\geq -R_{n}
\]%
so%
\[
V_{n}(T_{0})+C_{n}(T_{0})\geq I_{n}(T_{0})-R_{n}.
\]%
At instantaneos default of the HF, each PB $n$ sells its hedge portfolio
over the period $[\omega _{n}(T_{0}),\omega _{n}(T_{0})+\mu _{n}]$

The stock experiences additive (accross PBs) downward treand from our market
impact model over the time it is selling its portfolio.

We are interested in the final stock price at time $\omega _{0}(T_{0})+\mu
_{0}$ (as we are looking at this from PB $0$ perspective) but also the
average price received over the course of the unwind or the overall realized
proceeds

Actually more generally this is all symmetric so maybe we do not need to
distinguish PB $0$ so for each PB we calculate%
\[
\frac{\pi _{n}}{\mu _{n}}\int_{\omega _{n}(T_{0})}^{\omega _{n}(T_{0})+\mu
_{n}}S(t)~dt+C_{n}(T_{0})
\]%
where the forst term represents the total price received in an unwind of the
whole hedge position assuming the same number of shares are sold every day
over the MPOR $\mu _{n}$.

\section{OLD}

We assume that at time $t_{0}>0$ there is an external shock to the stock
price that results in the percentage drop of a fixed size $d,$ which is a
model input,%
\begin{gather*}
\frac{S(t_{0})-S(t_{0}-)}{S(t_{0}-)}=\delta , \\
S(t_{0})=S(t_{0}-)+\delta S(t_{0}-).
\end{gather*}

Recall that we consider everything from the point of view of PB $0.$ If 
\begin{eqnarray*}
\pi _{0}\left( S(t_{0})-S(0)\right) +V_{0}(t_{0})+R_{0} &>&0 \\
\pi _{0}\delta S(t_{0}-)+\pi _{0}\left( S(t_{0}-)-S(0)\right)
+V_{0}(t_{0})+R_{0} &>&0
\end{eqnarray*}%
Then nothing happens, as no margin call.

If, on the other hand, $\delta $ is sufficiently negative so that 
\[
\pi _{0}\delta S(t_{0}-)<-\pi _{0}\left( S(t_{0}-)-S(0)\right)
-V_{0}(t_{0})-R_{0}
\]%
there will be a margin call. The HF can either meet the margin call, or not.
Let us denote the probability of not meeting the margin call by $p,$ and we
condition the discussion below on this event, let's call the event $D$ (for
"default", noturally). PB $0$ will sell its hedge position from the time $%
t_{0}+\omega _{0}$ to time $t_{0}+\omega _{0}+\mu _{0}$ in equal amounts
each day (XXX this is or assumption; I suppose one can ask what is the
optimal execution strategy given everything that happens next)

Our main question is how much PB $0$ will be able to realize/its overall
loss given the dynamics we describe below.

The considerations above apply to all other PBs $n=1,\dots ,N.$ We assume
that the probability $p$ of missing a margin call is the same against all
PBs. The only difference is in reaction times of different PBs. The period
of interest to us is from $t_{0}$ to $t_{0}+\omega _{0}+\mu _{0}.$ Consider
PB $n.$ It will have its own margin call at time 
\[
d<
\]
we have $\omega _{n}$ which is the t

\end{document}
