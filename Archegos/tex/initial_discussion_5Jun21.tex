
\documentclass{article}
%%%%%%%%%%%%%%%%%%%%%%%%%%%%%%%%%%%%%%%%%%%%%%%%%%%%%%%%%%%%%%%%%%%%%%%%%%%%%%%%%%%%%%%%%%%%%%%%%%%%%%%%%%%%%%%%%%%%%%%%%%%%%%%%%%%%%%%%%%%%%%%%%%%%%%%%%%%%%%%%%%%%%%%%%%%%%%%%%%%%%%%%%%%%%%%%%%%%%%%%%%%%%%%%%%%%%%%%%%%%%%%%%%%%%%%%%%%%%%%%%%%%%%%%%%%%
%TCIDATA{OutputFilter=LATEX.DLL}
%TCIDATA{Version=5.50.0.2953}
%TCIDATA{<META NAME="SaveForMode" CONTENT="1">}
%TCIDATA{BibliographyScheme=Manual}
%TCIDATA{Created=Sunday, June 06, 2021 10:10:21}
%TCIDATA{LastRevised=Sunday, June 06, 2021 10:31:09}
%TCIDATA{<META NAME="GraphicsSave" CONTENT="32">}
%TCIDATA{<META NAME="DocumentShell" CONTENT="Standard LaTeX\Blank - Standard LaTeX Article">}
%TCIDATA{CSTFile=40 LaTeX article.cst}

\newtheorem{theorem}{Theorem}
\newtheorem{acknowledgement}[theorem]{Acknowledgement}
\newtheorem{algorithm}[theorem]{Algorithm}
\newtheorem{axiom}[theorem]{Axiom}
\newtheorem{case}[theorem]{Case}
\newtheorem{claim}[theorem]{Claim}
\newtheorem{conclusion}[theorem]{Conclusion}
\newtheorem{condition}[theorem]{Condition}
\newtheorem{conjecture}[theorem]{Conjecture}
\newtheorem{corollary}[theorem]{Corollary}
\newtheorem{criterion}[theorem]{Criterion}
\newtheorem{definition}[theorem]{Definition}
\newtheorem{example}[theorem]{Example}
\newtheorem{exercise}[theorem]{Exercise}
\newtheorem{lemma}[theorem]{Lemma}
\newtheorem{notation}[theorem]{Notation}
\newtheorem{problem}[theorem]{Problem}
\newtheorem{proposition}[theorem]{Proposition}
\newtheorem{remark}[theorem]{Remark}
\newtheorem{solution}[theorem]{Solution}
\newtheorem{summary}[theorem]{Summary}
\newenvironment{proof}[1][Proof]{\noindent\textbf{#1.} }{\ \rule{0.5em}{0.5em}}
\input{tcilatex}
\begin{document}


I do think it is interesting and has potential, both practically and as a
model exercise

Let's do it!

The good thing about the canvas you're summarising below is that it can
cover the full spectrum from the full uncertainty (we don't know either the
other PB's positions, or the other PB's margin arrangements), progressively
to the full certainty (we know both or have a tight distribution around
those). For the full uncertainty it would probably be a rather wide-band
calculation, so maybe not practically very useful, but this assessment could
be another way of driving home that the industry needs to have certainty, at
least about the positions if not about the specific margin arrangements.

Anton.

\TEXTsymbol{>} On 5 Jun 2021, at 11:29 am, Vladimir V. Piterbarg \TEXTsymbol{%
<}piterbarg@gmail.com\TEXTsymbol{>} wrote:

\TEXTsymbol{>}

\TEXTsymbol{>} \U{feff}

\TEXTsymbol{>} Privet

\TEXTsymbol{>}

\TEXTsymbol{>} was great to see you yesterday. We sort of joked about
writing a paper inspired by Archegos, but the more I think about it, the
more I see that we can possibly get something interesting going. Or at least
explore it. As a very initial brainstorming, this could go something like
this. The paper is about enhancing existing margin models with a concept of
some sort of feedback loop for large positions where, as you said, the MPOR
concept does not work

\TEXTsymbol{>}

\TEXTsymbol{>} 1. Existing margin models assume an orderly if slow unwind of
a position into a market that can absorb it without market impact

\TEXTsymbol{>} 2. Not realistic as we have seen for some large positions

\TEXTsymbol{>} 3. Let's say a PB has a concentrated large position with a
HF. The new model we are building would recognize that a similar position
(some distribution here) can be held with other PBs (assume the number of
PBs a HF has is known, or can be well approximated, or some random number)

\TEXTsymbol{>} 4. assume some distribution of margin held by different PBs
against the same position

\TEXTsymbol{>} 5. Build a feedback model that, as soon as the position
breaches the lowest-margin PB limit (or a margin call is missed), the PB
would start unwinding, thus triggering a feedback loop that moves the
position against better capitalized PBs who in turn start unwinding leading
to a cascade effect

\TEXTsymbol{>} 6. Then the distribution that should be used for margin would
have a drift from "rush to the exit" by PBs, and also some market volatility
as in a "normal" margin model

\TEXTsymbol{>} 7. We build some, perhaps somewhat complicated, model like
that and then, if we can, simplify it to some simple rule that, let's say,
can be offered to the industry as a reasonable rule of thumb to add on to
the "standard" margin for large concentrated positions

\end{document}
