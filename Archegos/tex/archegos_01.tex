
\documentclass{article}
%%%%%%%%%%%%%%%%%%%%%%%%%%%%%%%%%%%%%%%%%%%%%%%%%%%%%%%%%%%%%%%%%%%%%%%%%%%%%%%%%%%%%%%%%%%%%%%%%%%%%%%%%%%%%%%%%%%%%%%%%%%%%%%%%%%%%%%%%%%%%%%%%%%%%%%%%%%%%%%%%%%%%%%%%%%%%%%%%%%%%%%%%%%%%%%%%%%%%%%%%%%%%%%%%%%%%%%%%%%%%%%%%%%%%%%%%%%%%%%%%%%%%%%%%%%%
\usepackage{amsmath}

\setcounter{MaxMatrixCols}{10}
%TCIDATA{OutputFilter=LATEX.DLL}
%TCIDATA{Version=5.50.0.2953}
%TCIDATA{<META NAME="SaveForMode" CONTENT="1">}
%TCIDATA{BibliographyScheme=Manual}
%TCIDATA{Created=Saturday, June 05, 2021 09:00:47}
%TCIDATA{LastRevised=Wednesday, June 09, 2021 08:23:26}
%TCIDATA{<META NAME="GraphicsSave" CONTENT="32">}
%TCIDATA{<META NAME="DocumentShell" CONTENT="Standard LaTeX\Blank - Standard LaTeX Article">}
%TCIDATA{Language=American English}
%TCIDATA{CSTFile=40 LaTeX article.cst}

\newtheorem{theorem}{Theorem}
\newtheorem{acknowledgement}[theorem]{Acknowledgement}
\newtheorem{algorithm}[theorem]{Algorithm}
\newtheorem{axiom}[theorem]{Axiom}
\newtheorem{case}[theorem]{Case}
\newtheorem{claim}[theorem]{Claim}
\newtheorem{conclusion}[theorem]{Conclusion}
\newtheorem{condition}[theorem]{Condition}
\newtheorem{conjecture}[theorem]{Conjecture}
\newtheorem{corollary}[theorem]{Corollary}
\newtheorem{criterion}[theorem]{Criterion}
\newtheorem{definition}[theorem]{Definition}
\newtheorem{example}[theorem]{Example}
\newtheorem{exercise}[theorem]{Exercise}
\newtheorem{lemma}[theorem]{Lemma}
\newtheorem{notation}[theorem]{Notation}
\newtheorem{problem}[theorem]{Problem}
\newtheorem{proposition}[theorem]{Proposition}
\newtheorem{remark}[theorem]{Remark}
\newtheorem{solution}[theorem]{Solution}
\newtheorem{summary}[theorem]{Summary}
\newenvironment{proof}[1][Proof]{\noindent\textbf{#1.} }{\ \rule{0.5em}{0.5em}}
\input{tcilatex}
\begin{document}


\section{Motivation}

We build a simple model that estimates the impact of a forced unwind of
large concentrated positions by a number of prime brokers that face the same
defaulting counterparty simultaneously, taking into account, critically, the
market impact of such unwind. We consider this as an add-on on top of the
traditional Initial Margin (IM) model.

\section{Background}

A hedge fund (HF) enters synthetic positions referencing a risk factor $%
S=S(t)$ that, for simplicity, we call a stock, against a number of prime
brokers (PBs). For simplicity we assume the HF wants to be long. 

A typical derivatives position is a total return swap (TRS) where a HF would
receive total returns on $S$ in exchange for some funding rate. Derivatives
are often preferred to cash equity as they are not subject to the same
disclosure requirements as cash equity positions.

A PB would hedge its short position in $S$ on a TRS by buying the equivalent
amount of stock. While the HF is solvent, the position\ is "riskless" in the
sense that price changes on the PB's holdings of the stock are just passed
by the PB\ on to the client who bears all market risk. If, however, the HF
defaults for whatever reason, the TRS will terminate, and the PB\ will be
left with an unhedged position in the stock. 

Naturally, it will try to unwind its hedge by selling the stock. If the
movement in the stock make it so that the value is less than what the PB
paid for the initial stock purchase, it would realize a loss. 

Hence, a PB typically requires a margin, a form of collateral, from the HF.
There are generally two kinds of margin. One is the \emph{variation margin}
which ensures that the PB's stock value+variation margin is (approximately)
equal to the original price the PB paid to buy the stock. This can also be
seen/interpreted as the MTM\ on the TRS that the PB would lose should the HF
default (here and throughout we ignore MTM changes on the funding leg of the
TRS). 

The other kind is the \emph{initial margin} that is designed to cover, with
high probability, losses on the unwind of the stock position in orderly
fashion over a number of days (MPOR), should the HF default and the stock
move adversely over MPOR.

\section{Setup}

Let $S=S(t)$ be the stock. Let us assume the underlying annualized lognormal
volatility of the stock is $\sigma $. 

We assume that there are $N$ PBs and the position of the HF vs the $n$-th PB
is $\pi _{n}$ units of $S,$ with $\pi _{n}>0,$ $n=1,\dots ,N.$ That means
the PB's exposure to the HF is $-\pi _{n}S,$ and hence it is holding stock
as a hedge with the value of%
\begin{equation*}
\pi _{n}S(t)
\end{equation*}%
at time $t.$ Moreover, to acquire the hedge, the PB borrowed money (most
likely internally, but we record it for proper accounting)\ and holds a cash
position in the amount of the (negative of) initial stock purchase price:%
\begin{equation*}
-\pi _{n}S(0).
\end{equation*}%
Hence the PB's MTM on the hedge portfolio at time $t,$ is%
\begin{equation*}
\pi _{n}\left( S(t)-S(0)\right) .
\end{equation*}%
Again, under the assumption of negligible variation of the funding leg MTM,
this is the same as the negative of the MTM change on the TRS from the PB's
point of view.

We assume that each PB at time $t$ holds $V_{n}(t)$ in variational margin
and $I_{n}(t)$ in the initial margin, $n=1,\dots ,N.$ Under a strong margin
agreement the VM\ will be set to keep the position at zero MTM\ at each
point in time, i.e. by requiring that

\begin{equation}
V_{n}(t)+\pi _{n}\left( S(t)-S(0)\right) =0.  \label{eq:vm1}
\end{equation}%
The IM\ will be calculated as a high left-tail percentile of the future
distribution of $S$ over a specified MPOR. In reality (\ref{eq:vm1}) may
only hold approximately due to the presence of thresholds, lags in posting
the margin, etc. Similarly, IMs are dependent on a particular model that a
given PB uses and could also be negotiated up-front as simply a percentage
of notional, say. 

The goal of our modelling is to estimate market feedback effects and its
impact on a forced unwind from the point of view of one of the PBs. Each PB
obviously knows (or at least should!) how its own VM and IM are calculated,
but would have little idea about others'. Nor would it know the critical
parameters such as actual exposures of other PBs to the risk factor of
interest. With that in mind, we will eventually impose some reasonable
distributions on $V_{n}(t)$'s and $I_{n}(t)$'s. 

We will impose distribution on these parameters for all PBs, without
singling out any particular one as "the" one we care about. But when
specifying the distributions for the parameters we can obviously tighten the
distribution of any one of them to adopt its point of view on the
distributions.

With the intention of randomizing these later, we assume that $V_{n}(t)$'s
and $I_{n}(t)$'s are given for $n=1,\dots ,N.$

\section{The model, assumptions and motivation}

\subsection{Information on default}

An important element of our model is the assumption on how the default of HF
is revealed to the market, here meaning all relevant PBs. We assume that if
the HF defaults at time $t,$ all PBs know about it at the same time. One can
contemplate more elaborate mechanisms such as the $n$-th PB only finds out
about HF's inability to post margin when it actually makes a margin call,
which could be some time after $t$ due to how the margin agreement is
structured, e.g. with thresholds, grace periods, etc. We will leave these
complications for later research as our aim is to have a simple model that
captures main features of the situation we are trying to model.

\subsection{Default trigger}

IM is typically calculated on each day (or at least regularly)\ assuming
immediate default of the client (HF). We view our model as a sort of an
add-on to that so make the same assumption -- our calculations as detailed
below are on each day assuming immediate default of the HF for whatever
reason (that forces liquidation of stock hedges by all PBs that push the
stock lower thus leading to additional loss beyond the IM percentile loss in
"normal" unwind that we are trying to estimate).

It is of course likely that, with a large position, an initial downward move
in the stock price triggered by some exogenous factor led to margin calls on
the stock that led to the default of the HF. 

The "immediate exogenous default" concept is appealing as it corresponds to
the IM\ setup and hence directly comparable, and of course is rather simple
in concept. We probably need to think a bit more about this as possible
future enhancements. There are a couple of angles I can think of right now.
One, if the default was triggered by a fall in price of the stock, that drop
will also contribute to the overall loss of PBs. This is easy to accommodate
as we just assume that the stock has a downward jump at time $t$ (the time
of calculation, assumed to be the default time of the HF)%
\begin{equation}
\frac{S(t)-S(t-)}{S(t-)}=\delta \leq 0  \label{eq:deft1}
\end{equation}%
where $\delta $ is yet another model parameter.

Another angle here could be to try to estimate the size of the jump $\delta $
that would \emph{actually} \emph{trigger }the default of the HF. So the
setup could be somewhat different, and the basic question that the model is
answering would be "what if there is a jump $\delta $ at time $t$" rather
than "what if there is a default of the HF\ at time $t$". More
assumptions/unknown parameters would be required to estimate the jump that
would trigger a default of the HF, however. So I think we should leave this
for further research (?)

\subsection{Decision lag}

Another important characteristic of our model is the fact that there is a
lag between the realization that the HF is possibly insolvent (e.g. hearing
in the news that a margin call from some other PB was missed) and a decision
to unwind the position. The behavioral motivation here is the empirically
observed fact that for a large position with, presumably, an important
client, a PB would need to reach certain level of consensus internally, and
perhaps have various discussions with the HF, before getting required senior
sign-offs for the unwind. Some PBs are more decisive here while others may
have more internal layers to navigate. We model this "decision lag" as a
random time $\omega _{n}(t)$ for the $n$-th PB, interpreted as the time when
the unwind starts should the HF default at time $t.$ We consider this time
as stochastic because of inherent uncertainty of how long the decision would
take. Moreover, the parameters of this uncertainty are also uncertain due to
imperfect information about the speed of decision making in PBs [XXX\ is
this too complicated? random variable with random parameters? maybe we
simplify this later].

\subsection{Model for market impact}

The model for the stock in the "normal" conditions is of the form%
\begin{equation}
dS(t)/S(t)=\sigma ~dW(t).  \label{eq:mi1}
\end{equation}%
Note that we ignore the drift because of short timeliness involved. However,
we introduce another key part of our model and that is of market impact. The
standard model of market impact is the so-called square root law [many refs
here, see "articles" folder] that basically says that the impact (downward
from selling in our case) is proportional to the square root of the number
of shares sold vs. daily volume:%
\begin{equation}
\frac{S(u+dt)-S(u)}{S(u)}\sim \limfunc{sign}(Q(u))\times c\times \hat{\sigma}%
\sqrt{\frac{\left\vert Q(u)\right\vert }{\nu }},  \label{eq:mi2}
\end{equation}%
where $Q(u)$ is the number of shares transacted at $u$, $\nu $ is the daily
volume in shares, $\hat{\sigma}$ is the \textbf{daily} volatility and $c$
some constant empirically observed to be close to $1,$ and $dt$ is one day, $%
dt=1/250.$

Eventually we will combine (\ref{eq:mi1}) and (\ref{eq:mi2}) where, for each
day of selling, we combine the contributions from those PBs that are
actually selling in (\ref{eq:mi2}).

\subsection{MPOR and execution strategy}

For the initial version of the model we assume that, given HF default at
time $t,$ each PB would be selling its shares in equal daily amounts over
the time period 
\begin{equation*}
\lbrack \omega _{n}(t),\omega _{n}(t)+\mu _{n}]
\end{equation*}%
where $\mu _{n},$ $n=1,\dots ,N,$ is the MPOR for he $n$-th PB. The
collection $\{\mu _{n}\}$ is yet another input to the model that, from the
point of view of one of the PBs, is random as it is not known with certainty.

\subsection{Optimal execution}

In future developments we could explore the question of what is the optimal
execution strategy for a given PB given our model (and the uncertainty over
parameters). In the "standard" optimal liquidation of a large order type
problem, the optimal is VWAP [XXX? is this true, need to check], with the
optimal strategy balancing market impact vs speed of execution. The problem
would be more complicated here as there will be multiple competing PBs.

\subsection{Losses}

The total value realized by the $n$-th PB by selling its stock, assuming
uniform execution strategy, is given by

\begin{equation*}
\frac{\pi _{n}}{\mu _{n}}\int_{\omega _{n}(t)}^{\omega _{n}(t)+\mu
_{n}}S(t)~dt.
\end{equation*}%
The loss, assuming the HF defaults at $t,$ is then defined as%
\begin{equation}
L_{n}=\frac{\pi _{n}}{\mu _{n}}\int_{\omega _{n}(t)}^{\omega _{n}(t)+\mu
_{n}}S(t)~dt+V_{n}(t)+I_{n}(t)-\pi _{n}S(0).  \label{eq:loss1}
\end{equation}

\section{The model, inner simulation}

Given the above assumptions and definitions, it is pretty obvious how we
would calculate the distribution of losses. 

\begin{enumerate}
\item There will be an outer loop to simulate the values of various
parameters (positions, VMs, IMs, MPORs, decision lags, etc.)

\item For each realization of the model parameters:

\item Assume immediate default of the HF. If using (\ref{eq:deft1}),
calculate $S(t)$ from the observed $S(t-),$ and assume all IMs, VMs etc. are
associated with $S(t-),$ thus baking in a degree of immediate loss into the
model

\item For each day $u$ after the default until the last $\max_{n=1,\dots
,N}\left\{ \omega _{n}(t)+\mu _{n}\right\} $:

\begin{enumerate}
\item Figure out the total amount of stock being sold:%
\begin{equation*}
Q(u)=\sum_{n=1}^{N}1_{\left\{ u\in \lbrack \omega _{n}(t),\omega _{n}(t)+\mu
_{n}]\right\} }\frac{\pi _{n}}{\mu _{n}}
\end{equation*}

\item Simulate the stock price to next day using (here $dt=1$ day $=1/250$) 
\begin{equation}
S(u+dt)=S(u)\left( 1-c\sigma \sqrt{\frac{Q(u)}{\nu }dt}+\sigma ~\zeta (u)%
\sqrt{dt}\right) .  \label{eq:S1}
\end{equation}%
Here as we recall $\sigma $ is the annual volatility, $c$ is an empirical
constant close to one (model parameter)$,$ $\nu $ is estimated daily volume
in shares, and $\zeta (u)$'s are i.i.d. Gaussian. 
\end{enumerate}

\item After each simulated path of $S$ is calculated, each PB's losses are
calculated using (\ref{eq:loss1}).

\item By running multiple paths we build a loss distribution for each PB

\item We can also calculate the loss under the standard IM\ model for
comparison by getting rid of the drift term in (\ref{eq:S1})
\end{enumerate}

\begin{remark}
As written, the model requires both $S(0)$ (in (\ref{eq:loss1})) and $S(t-)$
(in (\ref{eq:deft1}), (\ref{eq:S1})). On the other hand, assuming perfect
collateralization with the variation margin, we would have%
\begin{equation}
V_{n}(t)+\pi _{n}\left( S(t-)-S(0)\right) =0  \label{eq:vm2}
\end{equation}%
so we can eliminate $S(0)$ from (\ref{eq:loss1}) and write%
\begin{equation*}
L_{n}=\frac{\pi _{n}}{\mu _{n}}\int_{\omega _{n}(t)}^{\omega _{n}(t)+\mu
_{n}}S(t)~dt+I_{n}(t)-\pi _{n}S(t-).
\end{equation*}
\end{remark}

\begin{remark}
The assumption (\ref{eq:vm2}) could be too strong, and we can replace it
with 
\begin{equation}
V_{n}(t)+\pi _{n}\left( S(t-)-S(0)\right) =G_{n}(t),  \label{eq:gap1}
\end{equation}%
where the "gap" $G_{n}(t)$ could be yet another model parameter, generally
bounded by the threshold $R_{n}$ of the $n$-th PB\ margin account,%
\begin{equation*}
\left\vert G_{n}(t)\right\vert \leq R_{n}
\end{equation*}%
for $n=1,\dots ,N.$ The last equation could be used in
simulating./randomizing these gaps. With (\ref{eq:gap1}) in mind, $S(0)$ can
be eliminated from (\ref{eq:loss1}) so that we can be write%
\begin{equation*}
L_{n}=\frac{\pi _{n}}{\mu _{n}}\int_{\omega _{n}(t)}^{\omega _{n}(t)+\mu
_{n}}S(t)~dt+I_{n}(t)+G_{n}(t)-\pi _{n}S(t-).
\end{equation*}
\end{remark}

\section{The model, outer simulation}

Here we need to think about how to parameterize distributions of various
model parameters in a reasonable and flexible way. TBC

\end{document}
